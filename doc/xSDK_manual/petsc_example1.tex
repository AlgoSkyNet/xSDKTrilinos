Our first example (Program \ref{PETSc_AnasaziEx.cpp}) shows how to compute
the smallest eigenpairs of a PETSc Mat, specificially Poisson2D, using Trilinos'
Anasazi package.

\begin{lstinputlisting}[caption=PETSc\_AnasaziEx.cpp,label=PETSc_AnasaziEx.cpp]{src/PETSc_AnasaziEx.cpp}
\end{lstinputlisting}

\paragraph{Lines 1--7}
Include statements

\paragraph{Lines 11--28}
Typedefs and using statements to make the code more readable

\paragraph{Lines 30--36}
PETSc variables

\paragraph{Lines 38--41}
Get the command line arguments using PETSc.  This example has three of them: the
number of mesh points in the x direction, number of mesh points in the y
direction, and the number of desired eigenpairs.

\paragraph{Lines 43--62}
Create the PETSc Mat and set its values.

\paragraph{Line 64}
Wrap the PETSc Mat in a Tpetra::PETScAIJMatrix.  Since Anasazi only requires a
Tpetra::Operator\footnote{RowMatrix is a specific type of Operator, and
CrsMatrix is a specific type of RowMatrix.  Therefore, you can use a RowMatrix
anywhere an Operator is accepted, but you can't necessarily use a RowMatrix
anywhere a CrsMatrix is expected.}, we do not have to deep copy the data to a
Tpetra::CrsMatrix.

\paragraph{Lines 66--67}
Create a random initial guess for the eigensolver.  Note that we can treat
tpetraA just like any other Tpetra::RowMatrix and obtain its domain map via
getDomainMap().

\paragraph{Lines 69--72}
Create the eigenproblem for Anasazi.  We provide the operator $A$ as well as our
initial guess for the set of desired eigenvectors to the constructor.  We then
request a certain number of eigenvectors and inform the eigensolver that our
problem is Hermitian\footnote{Some eigensolvers are optimized for use on
Hermitian eigenproblems.  Others do not work on non-Hermitian problems at
all, so it is important to specify this.}.

\paragraph{Lines 74--77}
Create the parameter list for the Riemannian Trust Region eigensolver.  We have
elected to have the solver print out the list of approximate eigenvalues at each
iteration, along with their associated residuals.  We also set our convergence
tolerance here.

\paragraph{Lines 79--82}
Solve the eigenvalue problem.  After it is solved, we may grab the eigenvalues
and eigenvectors via getSolution().
