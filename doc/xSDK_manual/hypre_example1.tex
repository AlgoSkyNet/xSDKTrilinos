In our first example (Program \ref{Hypre_SolveEx.cpp}), we will examine how
to use hypre solvers and preconditioners with Tpetra objects.

\begin{lstinputlisting}[caption=Hypre\_SolveEx.cpp,label=Hypre_SolveEx.cpp]{src/Hypre_SolveEx.cpp}
\end{lstinputlisting}

\paragraph{Lines 1--5}
Include statements

\paragraph{Lines 8--24}
Typedefs and using statements to make the code more readable

\paragraph{Lines 26--30}
Set up MPI

\paragraph{Lines 32--40}
Parse command line arguments.  This program allows the user to specify how large
the problem should be and how accurately the linear system should be solved.

\paragraph{Lines 42--76}
Set up the 2D Laplace operator.

\paragraph{Lines 78--81}
Create a random right-hand-side and initialize the solution vector to 0.

\paragraph{Lines 83--94}
Set hypre options (documented in the hypre user and reference manuals found at
\url{http://computation.llnl.gov/project/linear_solvers/software.php}). In this
example, we have elected to use the conjugate gradient method with an algebraic
multigrid preconditioner.  We have specified a particular coarsening and
relaxation type.  The most important thing to note about BoomerAMG is that if
you would like to use it as a preconditioner, you must set its maximum number of
iterations to 1; otherwise, hypre will assume you meant to use it as a linear
solver.  We then set the tolerance and maximum number of iterations for hypre's
conjugate gradient solver.

\paragraph{Lines 96--106}
Create the hypre solver.  Line 99 specifies that we will be using a hypre linear
solver, and line 100 says it will be the conjugate gradient method.  Line 101
says we would also like to use BoomerAMG.  Remember that you must also set
``SetPreconditioner'' to true, or the preconditioner will not be used.  Lines
103 and 104 specify the hypre parameters such as print level, tolerance, and
maximum number of iterations.

\paragraph{Lines 108--109}
Solve the linear system.  The function ``apply'' actually calls the hypre linear
solve routine PCG with a BoomerAMG preconditioner, as we specified above.