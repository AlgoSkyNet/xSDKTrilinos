\chapter{Installing Trilinos with the xSDK interfaces enabled}
\label{install_appendix}
In general, we recommend using the xSDK installation script provided at
\url{http://xsdk.info/getting-started/}.  If you would prefer to manually
install Trilinos with the xSDK interfaces enabled, you may use the following
procedure.

\begin{enumerate}
  \item Download and install the external packages (such as PETSc, hypre, and
  SuperLU)
  \item Go to the packages subdirectory of Trilinos and clone the xSDKTrilinos
        repository \\
        {\tt \$ cd /path/to/Trilinos/packages}
  \item Clone the xSDKTrilinos repository \\
        {\tt \$ git clone https://github.com/trilinos/xSDKTrilinos.git}
  \item Go to the directory where you wish to build Trilinos and invoke your
        configuration script. \\
        {\tt \$ cd ~/trilinos-build
             \$ ./do-configure \\
             \$ make -j8 \\
             \$ make install \\}
        An example script is presented as Program \ref{config-script}.
        
        The first thing we do in the example script is
        define where the Trilinos source code is located.  Then, we set the
        paths to the include directory and library directory for each third
        party library (TPL) we wish to enable.  Here, we have decided to enable
        PETSc, hypre, and SuperLU\_Dist.  Since SuperLU\_Dist requires METIS and
        ParMETIS, we must enable them as well.  The first couple of arguments
        to CMake are standard Trilinos arguments documented in the Trilinos
        Configure, Build, Test, and Install Reference Guide (found at
        \url{https://trilinos.org/docs/files/TrilinosBuildReference.html}).
        TPLs are addressed in in section 5.13 of that manual. 
        For each TPL we wish to enable, we add {\tt -D 
        TPL\_ENABLE\_[TPL\_NAME]}, {\tt -D [TPL\_NAME]\_LIBRARY\_DIRS}, and {\tt
        -D [TPL\_NAME]\_INCLUDE\_DIRS}.  We explicitly enable the packages
        Amesos2 and xSDKTrilinos because they contain the interfaces to
        SuperLU\_Dist, hypre, and PETSc.
  \item You may test your install by invoking ctest in your build directory \\
        {\tt \$ ctest}
\end{enumerate}

\begin{lstinputlisting}[caption=Sample
configuration script,label=config-script]{src/do-configure}
\end{lstinputlisting}