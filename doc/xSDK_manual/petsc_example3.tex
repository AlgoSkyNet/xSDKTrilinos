The following example (Program \ref{Tpetra_KSPEx.cpp}) illustrates this process
in greater detail.  Note that this example does not contain a single of PETSc
code.

\begin{lstinputlisting}[caption=Tpetra\_KSPEx.cpp,label=Tpetra_KSPEx.cpp]{src/Tpetra_KSPEx.cpp}
\end{lstinputlisting}

\paragraph{Lines 1--12}
Include statements

\paragraph{Lines 15--22}
Typedefs and using statements to make the code more readable

\paragraph{Lines 25--29}
Set up MPI and get the default communicator.

\paragraph{Lines 32--43} 
Parse command line arguments.  This program allows the user to specify the
filename for the matrix, the tolerance for the linear solve, the maximum number
of iterations, and the number of right hand sides for the linear system.

\paragraph{Lines 46--54}
Set up the linear system by reading the matrix from a file, creating a
random solution and setting the right hand side accordingly.  The initial
guess for the solution is set to $\vec{0}$.

\paragraph{Lines 57--63}
Set up the Ifpack2 Jacobi preconditioner.

\paragraph{Lines 66--70}
Set the Belos parameters via a Teuchos::ParameterList. We set the maximum number
of iterations, convergence tolerance, and which PETSc KSP solver is being used. 
Here we chose BiCGStab, but a list of all valid linear solver options can be
found at \url{http://www.mcs.anl.gov/petsc/petsc-current/docs/manualpages/KSP/KSPType.html}.
We also set the verbosity to IterationDetails, meaning PETSc will print the
residual norm at each iteration.

\paragraph{Lines 73--76}
Set up the linear problem for the Belos solver.

\paragraph{Lines 79--80}
Create the linear solver.  Note that even though PETSc is being used to solve
the linear system, you construct and use a Belos Solver Manager as you would for
any of the native Krylov solvers.

\paragraph{Line 83}
Solve the linear system.  The solution will overwrite the initial guess vector
$X$.
