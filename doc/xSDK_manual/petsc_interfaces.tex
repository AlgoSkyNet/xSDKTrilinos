\section{Trilinos-PETSc}
There is a two-way interface between PETSc and Trilinos which allows users to
use PETSc datatypes with Trilinos and vice-versa.

\subsection{Using PETSc Mat and Vec with Trilinos solvers}
Trilinos has two new interfaces to support using PETSc Mat anywhere
a Tpetra::Row\-Matrix or Tpetra::CrsMatrix can be used.  For packages requiring
a Tpetra::RowMatrix or Tpetra::Operator, such as Anasazi and Belos, you may wrap a
PETSc Mat in a Tpetra::PETSc\-AIJ\-Matrix; otherwise, you can copy it to a
Tpetra::CrsMatrix.  We will demonstrate each of those functions in the examples
below.

Our first example (Program \ref{PETSc_AnasaziEx.cpp}) shows how to compute
the smallest eigenpairs of a PETSc Mat, specificially Poisson2D, using Trilinos'
Anasazi package.

\begin{lstinputlisting}[caption=PETSc\_AnasaziEx.cpp,label=PETSc_AnasaziEx.cpp]{src/PETSc_AnasaziEx.cpp}
\end{lstinputlisting}

\paragraph{Lines 1--7}
Include statements

\paragraph{Lines 11--28}
Typedefs and using statements to make the code more readable

\paragraph{Lines 30--36}
PETSc variables

\paragraph{Line 39}
Initialize PETSc.

\paragraph{Lines 42--62}
Create the PETSc Mat and set its values.

\paragraph{Line 65}
Wrap the PETSc Mat in a Tpetra::PETScAIJMatrix.  Since Anasazi only requires a
Tpetra::Operator\footnote{RowMatrix is a specific type of Operator, and
CrsMatrix is a specific type of RowMatrix.  Therefore, you can use a RowMatrix
anywhere an Operator is accepted, but you can't necessarily use a RowMatrix
anywhere a CrsMatrix is expected.}, we do not have to deep copy the data to a
Tpetra::CrsMatrix.

\paragraph{Lines 68--69}
Create a random initial guess for the eigensolver.  Note that we can treat
tpetraA just like any other Tpetra::RowMatrix and obtain its domain map via
getDomainMap().

\paragraph{Lines 72--75}
Create the eigenproblem for Anasazi.  We provide the operator $A$ as well as our
initial guess for the set of desired eigenvectors to the constructor.  We then
request a certain number of eigenvectors and inform the eigensolver that our
problem is Hermitian\footnote{Some eigensolvers are optimized for use on
Hermitian eigenproblems.  Others do not work on non-Hermitian problems at
all, so it is important to specify this.}.

\paragraph{Lines 78--80}
Create the parameter list for the Riemannian Trust Region eigensolver.  We have
elected to have the solver print out the list of approximate eigenvalues at each
iteration, along with their associated residuals.  We also set our convergence
tolerance here.

\paragraph{Lines 83--98}
Solve the eigenvalue problem.  After it is solved, we may grab the eigenvalues
and eigenvectors via getSolution().

\paragraph{Line 101}
Terminate PETSc.


The second example (Program \ref{PETSc_Amesos2Ex.cpp}) demonstrates how to use
PETSc datatypes with Trilinos packages that require a Tpetra::CrsMatrix.  One
such package is Amesos2, which contains a variety of direct solvers (and
interfaces to external direct solvers).  In this example, we will solve a linear
system $Ax=b$ where $A$ is the 2D discretization of the Poisson operator on a
unit cube, and $b$ is a random vector.

\begin{lstinputlisting}[caption=PETSc\_Amesos2Ex.cpp,label=PETSc_Amesos2Ex.cpp]{src/PETSc_Amesos2Ex.cpp}
\end{lstinputlisting}

\paragraph{Lines 1--65}
These lines are very similar to the previous example.  We set up convenient
typedefs, read in the command line arguments with PETSc (in this case, the x
and y dimensions of the grid), and set up the Poisson2D matrix.

\paragraph{Lines 67--73}
Create the initial guess for $x$ and the random RHS $b$.

\paragraph{Lines 75--77}
Copy the PETSc Mat and Vecs to Tpetra::CrsMatrix and Tpetra::Vector, so that
we can use them with the Amesos2 linear solvers.

\paragraph{Lines 79--82}
Create an Amesos2 linear solver, specifically the native solver KLU2.  Perform
the symbolic and numeric factorizations, then solve the linear system.


\subsubsection{Is the data copied or wrapped?}
If you are using a part of Trilinos that requires Operator or RowMatrix, the
data is wrapped.  If you need a CrsMatrix specifically, the data is deep-copied.

\subsection{Using Trilinos datatypes with PETSc KSP solvers}
If you would like to use Trilinos datatypes, such as Tpetra::Operator and
Tpetra::Multi\-Vector, with a PETSc KSP linear solver, you may use the new
Belos\footnote{Belos is the iterative solver package of Trilinos.} interface:
PETScSolMgr.  This interface is very similar to that of the other native Belos
linear solvers, which makes solving linear systems such as $AX=B$ a simple
process.

\begin{enumerate}
  \item (Optional) Create a Tpetra::Operator for the preconditioner $M \approx
  A$.  You may use the preexisting preconditioners of Ifpack2 and MueLu, or you
  may create your own custom preconditioner.  Alternatively, you may choose not
  to use a preconditioner at all.
  \item Create a Belos::LinearProblem containing the operator $A$, the initial
  guess $X$, the right-hand side $B$, and the preconditioner $M$ (if you have
  one).
  \item Create a Teuchos::ParameterList containing the parameters you wish to
  set.  These parameters are summarized in Table \ref{table:ksp_parameters}.
  \item Create a Belos::PETScSolMgr with the LinearProblem and ParameterList
  from the previous steps.
  \item Call solve()
\end{enumerate}

\begin{table}
\center
\begin{tabular}{p{.9in} p{3.5in} c}
  \hline
  Parameter & Description & Default Value \\
  \hline
  Maximum Iterations & integer defining the maximum number of iterations to be
  performed. & 1000 \\
  \hline
  Solver & string defining the linear solver to be used. A list of all valid
  linear solver options can be found at
  \url{http://www.mcs.anl.gov/petsc/petsc-current/docs/manualpages/KSP/KSPType.html}
  & KSPGMRES \\
  \hline
  Verbosity & Belos::MsgType defining the amount of output the program should
  produce. Options include Belos::Errors, Belos::Warnings,
  Belos::IterationDetails, Belos::TimingDetails, and Belos::StatusTestDetails &
  Belos::Errors
  \\
  \hline
  Convergence Tolerance & double defining the tolerance of the linear solver &
  $10^{-8}$
  \\
  \hline
\end{tabular}
\caption{Belos::PETScSolMgr parameters}
\label{table:ksp_parameters}
\end{table}

The following example (Program \ref{Tpetra_KSPEx.cpp}) illustrates this process
in greater detail.  Note that this example does not contain a single of PETSc
code.

\begin{lstinputlisting}[caption=Tpetra\_KSPEx.cpp,label=Tpetra_KSPEx.cpp]{src/Tpetra_KSPEx.cpp}
\end{lstinputlisting}

\paragraph{Lines 1--9}
Include statements

\paragraph{Lines 12--24}
Typedefs and using statements to make the code more readable

\paragraph{Lines 26--29}
Set up MPI

\paragraph{Lines 31--40} 
Parse command line arguments.  This program allows the user to specify the
filename for the matrix, the tolerance for the linear solve, and which linear
solver is used.  A list of all valid linear solver options can be found at
\url{http://www.mcs.anl.gov/petsc/petsc-current/docs/manualpages/KSP/KSPType.html}.

\paragraph{Lines 42--46}
Set up the linear system by reading the matrix from a file, setting the initial
guess for the solution as $\vec{0}$ and setting the right hand side as
$\vec{1}$.

\paragraph{Lines 48--54}
Set up the Ifpack2 Jacobi preconditioner.

\paragraph{Lines 56--59}
Set the maximum number of iterations, convergence tolerance, and which PETSc
KSP solver is being used.

\paragraph{Lines 61--64}
Set up the linear problem for the Belos solver.

\paragraph{Lines 66-68}
Solve the linear system.

\subsubsection{Can I use this to solve linear systems with multiple
right-hand sides?}
Yes.  Unfortunately, PETSc has no support for multivectors at
this time, so each of the right hand sides will be processed independently.
If you want block or pseudo-block linear solvers, those are available within
Trilinos.

\subsubsection{Is the data copied or wrapped?}
The raw Tpetra matrix (or operator) data is wrapped rather than deep copied. 
The same applies to the preconditioner, if you are using a preconditioner.
